\documentclass[a4paper,12pt]{article}  
\usepackage[UTF8]{ctex}  
\usepackage{xcolor}
\usepackage{float}
\usepackage{graphicx} % Required for inserting images  
\usepackage{amsmath}
\begin{document}  
\begin{figure}
    \centering
    \includegraphics[width=0.5\linewidth]{ouc.png}
    \label{fig:enter-label}
\end{figure}
 \title{实验报告}  
 \author{王桂鑫}  
 \date{\today}  
 \maketitle  
GitHub地址:
\section{\color{blue}实验要求}  
熟练掌握版本控制(git)和LaTeX文档编译。  

\section{\color{red}实验过程与结果}  
\subsection{git} 
\subsubsection{git安装与环境配置}
\begin{figure}[H]
    \centering
    \includegraphics[width=0.5\linewidth]{屏幕截图 2024-08-30 010428.png}
    \caption{Enter Caption}
    \label{fig:git官网}
\end{figure}
\subsubsection{git基础命令}
1. \texttt{git config --global user.name "your name"} \\
  此命令用于设置您的用户名。\\

2. \texttt{git config --global user.email "your email"} \\
  此命令用于设置您的电子邮件。\\

3. \texttt{git config --global core.editor emacs} \\
  将编辑器更改为Emacs。\\

4. \texttt{git config -l} \\
  列出当前设置的所有配置参数。\\

5. \texttt{git init} \\
  初始化一个文件夹。\\

6. \texttt{git clone https://github.com/OI-wiki/OI-wiki} \\
  使用HTTP(S)连接远程仓库。\\

7. \texttt{git status} \\
  查看当前仓库文件的状态。\\

8. \texttt{git add <file>} \\
  将指定文件包括在版本跟踪中。\\

9. \texttt{git commit} \\
  提交更改。\\

10. \texttt{git log} \\
  查看提交历史。\\

11. \texttt{git branch} \\
  创建分支。\\

12.\texttt{git switch}\\
  切换分支。\\

13.\texttt{git merge}\\
  将该分支合并到当前分支上。\\

14.\texttt{git branch -d dev}\\
  对于未合并的分支,可以使用-D参数强制删除。\\

15.\texttt{git merge <branch> --squash}\\
  使用Squash方式进行分支合并。\\

16.\texttt{git rebase}\\
  将分支并入。\\

17.\texttt{git remote}\\
  查看当前仓库的远程仓库列表。\\

18.\texttt{git remote add <name> <url>}\\
  添加一个名字为name,链接为url的远程仓库。 \\

19.\texttt{git remote rename <oldname> <newname>}\\
  将名字为oldname的远程仓库改名为newname。\\

20.\texttt{git remote rm<name>}\\
  可以删除名字为name的远程仓库。\\

21.\texttt{git remote get-url<name>}\\
  查看名字为name的远程仓库链接。\\

22.\texttt{git remote set-url <name> <newurl>}\\
  将名字为name的远程仓库的链接改名为newurl。\\

23.\texttt{git fetch}\\
  获取远程仓库的更改。\\

24.\texttt{git pull}\\
  将远程更改合并。\\

25.\texttt{git push}\\
  将更改推送到远程仓库。\\

26.\texttt{git clone git@github.com:OI-wiki/OI-wiki.git}\\
  通过ssh连接到远程仓库。\\

\subsection{LaTeX}
\subsubsection{LaTex安装与环境配置}
\begin{figure}[H]
    \centering
    \includegraphics[width=0.5\linewidth]{屏幕截图 2024-08-30 083541.png}
    \caption{清华大学开源软件网址}
    \label{fig:enter-label}
\end{figure}
\subsubsection{\color{green}基础命令}
使用基础命令设计实验报告模板
\subsubsection{\color{yellow}结构}
\\1.documentclass\\
2.usepackage\\
3.chapter--章\\
4.section--节\\
5.subsection--小节\\
6.subsubsection--小小节\\
7.par--分段\\
8.newpage--分页命令\\
\subsubsection{插入}
\item 数学公式
          \begin{itemize}
              \item 正文行中的特殊字符和短公式:使用两个\$包括要表达的公式
              \item 单行公式带编号:equation
              \item 无编号公式:使用双\$包括
              \item 多行公式:split(usepackage(amsmath))
              \item 多情况讨论:cases(usepackage(amsmath))
              \begin{equation}
                F(x)=
                \begin{cases}
                    0&,\color{blue}\text{if $x=0$}\\
                    -x+1&, \color{blue}\text{if $x>0$}\\
                    x+1&, \color{blue}\text{if $x<0$}
                \end{cases}
              \end{equation}
          \end{itemize}
\item 图片
          \begin{itemize}
              \item usepackage\{graphicx\}
              \item begin\{figure\}...end\{figure\}\\begin\{figure*\}...end\{figure*\}
              \item 常用命令:
                    \begin{itemize}
                        \item centering:居中
                        \item includegraphics[图片大小][图片路径]
                        \item caption\{图片说明\}
                        \item label\{标签\}
                    \end{itemize}
              \item htbp
                    \begin{itemize}
                        \item h(here):尽量放置在代码所在位置
                        \item t(top):放置在页面顶部
                        \item b(bottom):放置在页面底部
                        \item p(page):单独放置在一个页面
                    \end{itemize}
          \end{itemize}
\item 表格\\选择在线生成工具:latex-tables.com
\item 文献引用
          \begin{itemize}
              \item cite\{lable\}
              \item begin\{thebibliography\}\\bibitem\{lable1\}...\\bibitem\{lable2\}\\end\{thebibliography\}
              \item BIBTex管理文献
                    \begin{enumerate}
                        \item 设定区域:bibliographystyle\{unsrt\}
                        \item bib文件:@article\{title,...\}
                        \item 插入位置:cite\{title\}
                        \item 参考文献位置:bibliography\{bib文件名\}
                    \end{enumerate}
          \end{itemize}
\end{document}
